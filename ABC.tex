\documentclass{article}\usepackage[]{graphicx}\usepackage[]{color}
%% maxwidth is the original width if it is less than linewidth
%% otherwise use linewidth (to make sure the graphics do not exceed the margin)
\makeatletter
\def\maxwidth{ %
  \ifdim\Gin@nat@width>\linewidth
    \linewidth
  \else
    \Gin@nat@width
  \fi
}
\makeatother

\definecolor{fgcolor}{rgb}{0.345, 0.345, 0.345}
\newcommand{\hlnum}[1]{\textcolor[rgb]{0.686,0.059,0.569}{#1}}%
\newcommand{\hlstr}[1]{\textcolor[rgb]{0.192,0.494,0.8}{#1}}%
\newcommand{\hlcom}[1]{\textcolor[rgb]{0.678,0.584,0.686}{\textit{#1}}}%
\newcommand{\hlopt}[1]{\textcolor[rgb]{0,0,0}{#1}}%
\newcommand{\hlstd}[1]{\textcolor[rgb]{0.345,0.345,0.345}{#1}}%
\newcommand{\hlkwa}[1]{\textcolor[rgb]{0.161,0.373,0.58}{\textbf{#1}}}%
\newcommand{\hlkwb}[1]{\textcolor[rgb]{0.69,0.353,0.396}{#1}}%
\newcommand{\hlkwc}[1]{\textcolor[rgb]{0.333,0.667,0.333}{#1}}%
\newcommand{\hlkwd}[1]{\textcolor[rgb]{0.737,0.353,0.396}{\textbf{#1}}}%

\usepackage{framed}
\makeatletter
\newenvironment{kframe}{%
 \def\at@end@of@kframe{}%
 \ifinner\ifhmode%
  \def\at@end@of@kframe{\end{minipage}}%
  \begin{minipage}{\columnwidth}%
 \fi\fi%
 \def\FrameCommand##1{\hskip\@totalleftmargin \hskip-\fboxsep
 \colorbox{shadecolor}{##1}\hskip-\fboxsep
     % There is no \\@totalrightmargin, so:
     \hskip-\linewidth \hskip-\@totalleftmargin \hskip\columnwidth}%
 \MakeFramed {\advance\hsize-\width
   \@totalleftmargin\z@ \linewidth\hsize
   \@setminipage}}%
 {\par\unskip\endMakeFramed%
 \at@end@of@kframe}
\makeatother

\definecolor{shadecolor}{rgb}{.97, .97, .97}
\definecolor{messagecolor}{rgb}{0, 0, 0}
\definecolor{warningcolor}{rgb}{1, 0, 1}
\definecolor{errorcolor}{rgb}{1, 0, 0}
\newenvironment{knitrout}{}{} % an empty environment to be redefined in TeX

\usepackage{alltt}
\usepackage[sc]{mathpazo}
\renewcommand{\sfdefault}{lmss}
\renewcommand{\ttdefault}{lmtt}
\usepackage[T1]{fontenc}
\usepackage{geometry}
\geometry{verbose,tmargin=2.5cm,bmargin=2.5cm,lmargin=2.5cm,rmargin=2.5cm}
\setcounter{secnumdepth}{2}
\setcounter{tocdepth}{2}
\usepackage[unicode=true,pdfusetitle,
 bookmarks=true,bookmarksnumbered=true,bookmarksopen=true,bookmarksopenlevel=2,
 breaklinks=false,pdfborder={0 0 1},backref=false,colorlinks=false]
 {hyperref}
\hypersetup{
 pdfstartview={XYZ null null 1}}

\makeatletter
%%%%%%%%%%%%%%%%%%%%%%%%%%%%%% User specified LaTeX commands.
\renewcommand{\textfraction}{0.05}
\renewcommand{\topfraction}{0.8}
\renewcommand{\bottomfraction}{0.8}
\renewcommand{\floatpagefraction}{0.75}

\makeatother
\IfFileExists{upquote.sty}{\usepackage{upquote}}{}
\begin{document}



\title{}



\maketitle
The results below are generated from an R script.

\begin{knitrout}
\definecolor{shadecolor}{rgb}{0.969, 0.969, 0.969}\color{fgcolor}\begin{kframe}
\begin{alltt}
\hlcom{#' Package: alhe}
\hlcom{#'}
\hlcom{#' @title Implementation of Artificial Bee Colony(ABC) algorithm}
\hlcom{#' for the purposes of ALHE class on Warsaw University of Technology}
\hlcom{#' @author Volodymyr Ostruk \textbackslash{}email\{ostruk@outlook.com\}}
\hlcom{#' Finds a local minimums of the function(aka "the best food sources")}
\hlcom{#' @references \textbackslash{}url\{https://en.wikipedia.org/wiki/Artificial_bee_colony_algorithm\}}
\hlcom{#' @references  Dervis KARABOGA; AN IDEA BASED ON HONEY BEE SWARM FOR NUMERICAL OPTIMIZATION; TECHNICAL REPORT-TR06, OCTOBER, 2005}
\hlcom{#'}
\hlcom{#' @param goal, f(x) - funkcja celu obliczana dla argumentu x}
\hlcom{#' @param dim, wymiarowosc dla ktorej prowadzone beda obliczenia}
\hlcom{#' @param pars, obiekt z pozostalymi parametrami metody:}
\hlcom{#'          - NP: ilosc pszczol,}
\hlcom{#'          - FoodNumber: ilosc zrodl jedzenia}
\hlcom{#'          - lb:  dolna granica zasiegu przeszukiwania - dla przyspieszenia}
\hlcom{#'          - ub:  gorna granica parametrow - dla przyspieszenia}
\hlcom{#'          - limit: zrodlo jedzenia, ktore nie moze byc ulepszone w tylu probach jest porzucane przez pszczoly}
\hlcom{#'          - maxCycle: maksymalna ilosc cykli(iteracji) - "stop" kryterium zatrzymania}
\hlcom{#'          - optbin:  TRUE jezeli chcemy optymalizowac binarnie [1,0]}
\hlcom{#'          - critical: ograniczenie na niezmiennosc rozwiazania - dla przyspieszenia}
\hlcom{#'}
\hlcom{#' @return res - wynik działania metody, przy czym:}
\hlcom{#'     - res$x to wektor odpowiadający najlepszemu rozwiązaniu,}
\hlcom{#'     - res$y to wartość f. celu dla tego wektora}
\hlcom{#'}
\hlcom{#' @examples}
\hlcom{#' abc(function(x) sum(x^2), 5, list(NP=40,lb=-100, ub=100, critical=100))}
\hlcom{#'}
\hlcom{#' @export}
\hlstd{abc} \hlkwb{<-} \hlkwa{function}\hlstd{(}\hlkwc{goal}\hlstd{,} \hlkwc{dim}\hlstd{,} \hlkwc{pars}\hlstd{)}
\hlstd{\{}

  \hlkwa{if}\hlstd{(}\hlopt{!}\hlstd{(}\hlstr{"NP"} \hlopt \hlkwd{names}\hlstd{(pars))) pars}\hlopt{$}\hlstd{NP} \hlkwb{<-} \hlnum{40}
  \hlkwa{if}\hlstd{(}\hlopt{!}\hlstd{(}\hlstr{"FoodNumber"} \hlopt \hlkwd{names}\hlstd{(pars))) pars}\hlopt{$}\hlstd{FoodNumber} \hlkwb{<-} \hlstd{pars}\hlopt{$}\hlstd{NP}\hlopt{/}\hlnum{2}
  \hlkwa{if}\hlstd{(}\hlopt{!}\hlstd{(}\hlstr{"lb"} \hlopt \hlkwd{names}\hlstd{(pars))) pars}\hlopt{$}\hlstd{lb} \hlkwb{<-} \hlopt{-}\hlnum{Inf}
  \hlkwa{if}\hlstd{(}\hlopt{!}\hlstd{(}\hlstr{"ub"} \hlopt \hlkwd{names}\hlstd{(pars))) pars}\hlopt{$}\hlstd{ub} \hlkwb{<-} \hlopt{+}\hlnum{Inf}
  \hlkwa{if}\hlstd{(}\hlopt{!}\hlstd{(}\hlstr{"limit"} \hlopt \hlkwd{names}\hlstd{(pars))) pars}\hlopt{$}\hlstd{limit} \hlkwb{=} \hlnum{100}
  \hlkwa{if}\hlstd{(}\hlopt{!}\hlstd{(}\hlstr{"maxCycle"} \hlopt \hlkwd{names}\hlstd{(pars))) pars}\hlopt{$}\hlstd{maxCycle} \hlkwb{=} \hlnum{1000}
  \hlkwa{if}\hlstd{(}\hlopt{!}\hlstd{(}\hlstr{"optbin"} \hlopt \hlkwd{names}\hlstd{(pars))) pars}\hlopt{$}\hlstd{optbin}\hlkwb{=}\hlnum{FALSE}
  \hlkwa{if}\hlstd{(}\hlopt{!}\hlstd{(}\hlstr{"critical"} \hlopt \hlkwd{names}\hlstd{(pars))) pars}\hlopt{$}\hlstd{critical}\hlkwb{=}\hlnum{50}

  \hlstd{pars}\hlopt{$}\hlstd{lb} \hlkwb{<-} \hlkwd{rep}\hlstd{(pars}\hlopt{$}\hlstd{lb, dim)}
  \hlstd{pars}\hlopt{$}\hlstd{ub} \hlkwb{<-} \hlkwd{rep}\hlstd{(pars}\hlopt{$}\hlstd{ub, dim)}
  \hlstd{pars}\hlopt{$}\hlstd{lb[}\hlkwd{is.infinite}\hlstd{(pars}\hlopt{$}\hlstd{lb)]} \hlkwb{<-} \hlopt{-}\hlstd{(.Machine}\hlopt{$}\hlstd{double.xmax}\hlopt{*}\hlnum{1e-10}\hlstd{)}
  \hlstd{pars}\hlopt{$}\hlstd{ub[}\hlkwd{is.infinite}\hlstd{(pars}\hlopt{$}\hlstd{ub)]} \hlkwb{<-} \hlopt{+}\hlstd{(.Machine}\hlopt{$}\hlstd{double.xmax}\hlopt{*}\hlnum{1e-10}\hlstd{)}
  \hlcom{# Initial params}
  \hlstd{Foods}       \hlkwb{<-} \hlkwd{matrix}\hlstd{(}\hlkwd{double}\hlstd{(pars}\hlopt{$}\hlstd{FoodNumber}\hlopt{*}\hlstd{dim),} \hlkwc{nrow}\hlstd{=pars}\hlopt{$}\hlstd{FoodNumber)}
  \hlstd{f}           \hlkwb{<-} \hlkwd{double}\hlstd{(pars}\hlopt{$}\hlstd{FoodNumber)}  \hlcom{#point - history wartosc f-ji celu dla danego rozwiazania}
  \hlstd{fitness}     \hlkwb{<-} \hlkwd{double}\hlstd{(pars}\hlopt{$}\hlstd{FoodNumber)}  \hlcom{#point - history quality of the point -}
  \hlstd{trial}       \hlkwb{<-} \hlkwd{double}\hlstd{(pars}\hlopt{$}\hlstd{FoodNumber)}  \hlcom{#point - history}
  \hlstd{prob}        \hlkwb{<-} \hlkwd{double}\hlstd{(pars}\hlopt{$}\hlstd{FoodNumber)}  \hlcom{#point - history}
  \hlstd{solution}    \hlkwb{<-} \hlkwd{double}\hlstd{(dim)}  \hlcom{#model}
  \hlstd{ObjValSol}   \hlkwb{<-} \hlkwd{double}\hlstd{(}\hlnum{1}\hlstd{)}
  \hlstd{FitnessSol}  \hlkwb{<-} \hlkwd{double}\hlstd{(}\hlnum{1}\hlstd{)}
  \hlstd{neighbour}   \hlkwb{<-} \hlkwd{integer}\hlstd{(}\hlnum{1}\hlstd{)}
  \hlstd{param2change}\hlkwb{<-} \hlkwd{integer}\hlstd{(}\hlnum{1}\hlstd{)}
  \hlstd{GlobalMin}   \hlkwb{<-} \hlkwd{double}\hlstd{(}\hlnum{1}\hlstd{)}      \hlcom{#model}
  \hlstd{GlobalParams}\hlkwb{<-} \hlkwd{double}\hlstd{(dim)}    \hlcom{#model}
  \hlcom{#GlobalMins  <- double(runtime)}
  \hlstd{r}           \hlkwb{<-} \hlkwd{integer}\hlstd{(}\hlnum{1}\hlstd{)}

  \hlcom{# Fun}
 \hlcom{# fun <- function(par) goal(par, ...)}

  \hlcom{# Fitness function - calculates quality}
  \hlstd{evaluation} \hlkwb{<-} \hlkwa{function}\hlstd{(}\hlkwc{f_wart}\hlstd{)}
  \hlstd{\{}
    \hlkwa{if} \hlstd{(f_wart} \hlopt{>=} \hlnum{0}\hlstd{)} \hlkwd{return}\hlstd{(}\hlnum{1}\hlopt{/}\hlstd{(f_wart} \hlopt{+} \hlnum{1}\hlstd{))}
    \hlkwa{else} \hlkwd{return}\hlstd{(}\hlnum{1} \hlopt{+} \hlkwd{abs}\hlstd{(f_wart))}
  \hlstd{\}}
  \hlcom{# evaluation(f[1])}

  \hlcom{# The best food source is memorized}
  \hlstd{MemorizeBestSource} \hlkwb{<-} \hlkwa{function}\hlstd{()}
  \hlstd{\{}
    \hlstd{change} \hlkwb{<-} \hlnum{0}
    \hlkwa{for}\hlstd{(i} \hlkwa{in} \hlkwd{seq}\hlstd{(}\hlnum{1}\hlstd{,pars}\hlopt{$}\hlstd{FoodNumber)) \{}
      \hlkwa{if} \hlstd{(f[i]} \hlopt{<} \hlstd{GlobalMin) \{}
        \hlstd{change} \hlkwb{<-} \hlstd{change} \hlopt{+} \hlnum{1}
        \hlstd{GlobalMin} \hlkwb{<<-} \hlstd{f[i]}

        \hlcom{# Replacing new group of parameters}
        \hlstd{GlobalParams} \hlkwb{<<-} \hlstd{Foods[i,]}
      \hlstd{\}}
    \hlstd{\}}
    \hlcom{# Increasing persistance}
    \hlkwa{if} \hlstd{(}\hlopt{!}\hlstd{change) p} \hlkwb{<<-} \hlstd{p} \hlopt{+} \hlnum{1}
  \hlstd{\}}

  \hlcom{# Variables are initialized in the range [pars$lb,pars$ub]. If each parameter has}
  \hlcom{# different range, use arrays pars$lb[j], pars$ub[j] instead of pars$lb and pars$ub}
  \hlcom{# Counters of food sources are also initialized in this function}

  \hlstd{init} \hlkwb{<-} \hlkwa{function}\hlstd{(}\hlkwc{index}\hlstd{,} \hlkwc{firstinit}\hlstd{=}\hlnum{FALSE}\hlstd{,} \hlkwc{...}\hlstd{) \{}
    \hlkwa{if} \hlstd{(pars}\hlopt{$}\hlstd{optbin) Foods[index,]} \hlkwb{<<-} \hlkwd{runif}\hlstd{(dim)} \hlopt{>} \hlnum{.5}
    \hlkwa{else} \hlstd{\{}
      \hlkwa{if} \hlstd{(}\hlopt{!}\hlstd{firstinit) \{}
        \hlstd{Foods[index,]} \hlkwb{<<-} \hlkwd{sapply}\hlstd{(}\hlnum{1}\hlopt{:}\hlstd{dim,} \hlkwa{function}\hlstd{(}\hlkwc{k}\hlstd{)} \hlkwd{runif}\hlstd{(}\hlnum{1}\hlstd{,pars}\hlopt{$}\hlstd{lb[k],pars}\hlopt{$}\hlstd{ub[k]) )} \hlcom{#uniform distr rand deriv}
      \hlstd{\}}
      \hlkwa{else} \hlstd{\{}
        \hlcom{# For the first initialization we set the bees at}
        \hlcom{# specific places equaly distributed through the}
        \hlcom{# bounds.}
        \hlstd{Foods[index,]} \hlkwb{<<-}
          \hlkwd{sapply}\hlstd{(}\hlnum{1}\hlopt{:}\hlstd{dim,} \hlkwa{function}\hlstd{(}\hlkwc{k}\hlstd{) \{}
            \hlkwd{seq}\hlstd{(pars}\hlopt{$}\hlstd{lb[k],pars}\hlopt{$}\hlstd{ub[k],}\hlkwc{length.out}\hlstd{=pars}\hlopt{$}\hlstd{FoodNumber)[index]}
          \hlstd{\}}
          \hlstd{)}
      \hlstd{\}}
    \hlstd{\}}

    \hlstd{solution} \hlkwb{<<-} \hlstd{Foods[index,]}

    \hlstd{f[index]} \hlkwb{<<-} \hlkwd{goal}\hlstd{(solution)}

    \hlstd{fitness[index]} \hlkwb{<<-} \hlkwd{evaluation}\hlstd{(f[index])}
    \hlstd{trial[index]} \hlkwb{<<-} \hlnum{0}

  \hlstd{\}}
  \hlcom{# init(2)}

  \hlcom{# All food sources are initialized}
  \hlstd{initialize} \hlkwb{<-} \hlkwa{function}\hlstd{(}\hlkwc{Foods}\hlstd{) \{}
    \hlkwd{sapply}\hlstd{(}\hlnum{1}\hlopt{:}\hlstd{pars}\hlopt{$}\hlstd{FoodNumber, init,} \hlkwc{firstinit}\hlstd{=}\hlnum{TRUE}\hlstd{)}

    \hlstd{GlobalMin} \hlkwb{<<-} \hlstd{f[}\hlnum{1}\hlstd{]}
    \hlstd{GlobalParams} \hlkwb{<<-} \hlstd{Foods[}\hlnum{1}\hlstd{,]}
    \hlkwd{return} \hlstd{(}\hlkwd{list}\hlstd{(Foods, fitness, trial, prob))}
  \hlstd{\}}

  \hlcom{# initial()}


  \hlstd{SendEmployedBees} \hlkwb{<-} \hlkwa{function}\hlstd{() \{}
    \hlkwa{for} \hlstd{(i} \hlkwa{in} \hlnum{1}\hlopt{:}\hlstd{pars}\hlopt{$}\hlstd{FoodNumber) \{}
      \hlstd{r} \hlkwb{<-} \hlkwd{runif}\hlstd{(}\hlnum{1}\hlstd{)}
      \hlcom{# The parameter to be changed is determined randomly}
      \hlstd{param2change} \hlkwb{<-} \hlkwd{floor}\hlstd{(r}\hlopt{*}\hlstd{dim)} \hlopt{+} \hlnum{1}

      \hlcom{# A randomly chosen solution is used in producing a mutant solution of the solution i}
      \hlstd{neighbour} \hlkwb{<-} \hlkwd{floor}\hlstd{(r}\hlopt{*}\hlstd{pars}\hlopt{$}\hlstd{FoodNumber)} \hlopt{+} \hlnum{1}

      \hlcom{# Randomly selected solution must be different from the solution i}
      \hlkwa{while}\hlstd{(neighbour}\hlopt{==}\hlstd{i)}
        \hlstd{neighbour} \hlkwb{<-} \hlkwd{floor}\hlstd{(}\hlkwd{runif}\hlstd{(}\hlnum{1}\hlstd{)}\hlopt{*}\hlstd{pars}\hlopt{$}\hlstd{FoodNumber)} \hlopt{+} \hlnum{1}

      \hlstd{solution} \hlkwb{<<-} \hlstd{Foods[i,]}

      \hlcom{# v_\{ij\}=x_\{ij\}+\textbackslash{}phi_\{ij\}*(x_\{kj\}-x_\{ij\})}
      \hlstd{r} \hlkwb{<-} \hlkwd{runif}\hlstd{(}\hlnum{1}\hlstd{)}

      \hlkwa{if} \hlstd{(pars}\hlopt{$}\hlstd{optbin) solution[param2change]} \hlkwb{<<-} \hlstd{r} \hlopt{>} \hlnum{0.5}
      \hlkwa{else} \hlstd{\{}
        \hlstd{solution[param2change]} \hlkwb{<<-}
          \hlstd{Foods[i,param2change]}\hlopt{+}
          \hlstd{(Foods[i,param2change]}\hlopt{-}\hlstd{Foods[neighbour,param2change])}\hlopt{*}\hlstd{(r}\hlopt{-}\hlnum{0.5}\hlstd{)}\hlopt{*}\hlnum{2}

        \hlcom{# if generated parameter value is out of boundaries, it is shifted onto the boundaries}
        \hlkwa{if} \hlstd{(solution[param2change]}\hlopt{<}\hlstd{pars}\hlopt{$}\hlstd{lb[param2change])}
          \hlstd{solution[param2change]}\hlkwb{<<-}\hlstd{pars}\hlopt{$}\hlstd{lb[param2change]}

        \hlkwa{if} \hlstd{(solution[param2change]}\hlopt{>}\hlstd{pars}\hlopt{$}\hlstd{ub[param2change])}
          \hlstd{solution[param2change]}\hlkwb{<<-}\hlstd{pars}\hlopt{$}\hlstd{ub[param2change]}
      \hlstd{\}}

      \hlstd{ObjValSol} \hlkwb{<<-} \hlkwd{goal}\hlstd{(solution)}
      \hlstd{FitnessSol} \hlkwb{<<-} \hlkwd{evaluation}\hlstd{(ObjValSol)}

      \hlcom{# a greedy selection is applied between the current solution i and its mutant*/}
      \hlkwa{if} \hlstd{(FitnessSol}\hlopt{>}\hlstd{fitness[i]) \{}
        \hlcom{# If the mutant solution is better than the current solution i, replace the solution with the mutant and reset the trial counter of solution i*/}
        \hlstd{trial[i]} \hlkwb{<<-} \hlnum{0}\hlstd{;}
        \hlcom{#for(j in 1:dim) Foods[i,j] <<- solution[j]}
        \hlstd{Foods[i,]} \hlkwb{<<-} \hlstd{solution}
        \hlstd{f[i]}\hlkwb{<<-} \hlstd{ObjValSol}
        \hlstd{fitness[i]}\hlkwb{<<-}\hlstd{FitnessSol}
      \hlstd{\}}
      \hlkwa{else} \hlstd{\{}
        \hlcom{# the solution i can not be improved, increase its trial counter*/}
        \hlstd{trial[i]} \hlkwb{<<-} \hlstd{trial[i]}\hlopt{+}\hlnum{1}
      \hlstd{\}}
    \hlstd{\}}
  \hlstd{\}}


  \hlcom{# A food source is chosen with the probability which is proportioal to its quality*/}
  \hlcom{# Different schemes can be used to calculate the probability values*/}
  \hlcom{# For example prob(i)=fitness(i)/sum(fitness)*/}
  \hlcom{# or in a way used in the metot below prob(i)=a*fitness(i)/max(fitness)+b*/}
  \hlcom{# probability values are calculated by using fitness values and normalized by dividing maximum fitness value*/}
  \hlstd{evaluateList} \hlkwb{<-} \hlkwa{function}\hlstd{(}\hlkwc{points}\hlstd{,}\hlkwc{evaluation}\hlstd{) \{}
    \hlstd{maxfit} \hlkwb{<-} \hlstd{fitness[}\hlnum{1}\hlstd{]}
    \hlkwa{for} \hlstd{(i} \hlkwa{in} \hlnum{1}\hlopt{:}\hlstd{pars}\hlopt{$}\hlstd{FoodNumber)}
      \hlkwa{if} \hlstd{(fitness[i]} \hlopt{>} \hlstd{maxfit) maxfit} \hlkwb{<-} \hlstd{fitness[i]}

      \hlstd{prob} \hlkwb{<<-} \hlnum{.9}\hlopt{*}\hlstd{(fitness}\hlopt{/}\hlstd{maxfit)} \hlopt{+} \hlnum{.1}
      \hlstd{prob[}\hlkwd{is.nan}\hlstd{(prob)]}  \hlkwb{<<-} \hlnum{.1}
      \hlkwd{return}\hlstd{(points)}
  \hlstd{\}}

  \hlstd{SendOnlookerBees} \hlkwb{<-} \hlkwa{function}\hlstd{()}
  \hlstd{\{}
    \hlcom{# Onlooker Bee phase}
    \hlstd{i} \hlkwb{<-} \hlnum{1}
    \hlstd{t} \hlkwb{<-} \hlnum{0}
    \hlkwa{while} \hlstd{(t} \hlopt{<} \hlstd{pars}\hlopt{$}\hlstd{FoodNumber)}
    \hlstd{\{}
      \hlstd{r} \hlkwb{<-} \hlkwd{runif}\hlstd{(}\hlnum{1}\hlstd{)}
      \hlcom{# choose a food source depending on its probability to be chosen}
      \hlkwa{if} \hlstd{(r} \hlopt{<} \hlstd{prob[i]) \{}
        \hlstd{t} \hlkwb{<-} \hlstd{t} \hlopt{+} \hlnum{1}
        \hlstd{r} \hlkwb{<-} \hlkwd{runif}\hlstd{(}\hlnum{1}\hlstd{)}

        \hlcom{# The parameter to be changed is determined randomly}
        \hlstd{param2change} \hlkwb{<-} \hlkwd{floor}\hlstd{(r}\hlopt{*}\hlstd{dim)} \hlopt{+} \hlnum{1}

        \hlcom{# A randomly chosen solution is used in producing a mutant solution of the solution i}
        \hlstd{neighbour} \hlkwb{<-} \hlkwd{floor}\hlstd{(r}\hlopt{*}\hlstd{pars}\hlopt{$}\hlstd{FoodNumber)} \hlopt{+} \hlnum{1}

        \hlcom{#Randomly selected solution must be different from the solution i*/}
        \hlkwa{while}\hlstd{(neighbour}\hlopt{==}\hlstd{i)}
          \hlstd{neighbour} \hlkwb{<-} \hlkwd{floor}\hlstd{(}\hlkwd{runif}\hlstd{(}\hlnum{1}\hlstd{)}\hlopt{*}\hlstd{pars}\hlopt{$}\hlstd{FoodNumber)} \hlopt{+} \hlnum{1}

        \hlstd{solution} \hlkwb{<<-} \hlstd{Foods[i,]}

        \hlstd{r} \hlkwb{<-} \hlkwd{runif}\hlstd{(}\hlnum{1}\hlstd{)}

        \hlkwa{if} \hlstd{(pars}\hlopt{$}\hlstd{optbin) solution[param2change]} \hlkwb{<<-} \hlstd{r} \hlopt{>} \hlnum{.5}
        \hlkwa{else}
        \hlstd{\{}
          \hlstd{solution[param2change]} \hlkwb{<<-}
            \hlstd{Foods[i,param2change]}\hlopt{+}
            \hlstd{(Foods[i,param2change]}\hlopt{-}\hlstd{Foods[neighbour,param2change])}\hlopt{*}\hlstd{(r}\hlopt{-}\hlnum{0.5}\hlstd{)}\hlopt{*}\hlnum{2}

          \hlcom{# if generated parameter value is out of boundaries, it is shifted onto the boundaries*/}
          \hlkwa{if} \hlstd{(solution[param2change]}\hlopt{<}\hlstd{pars}\hlopt{$}\hlstd{lb[param2change])}
            \hlstd{solution[param2change]} \hlkwb{<<-} \hlstd{pars}\hlopt{$}\hlstd{lb[param2change]}

          \hlkwa{if} \hlstd{(solution[param2change]}\hlopt{>}\hlstd{pars}\hlopt{$}\hlstd{ub[param2change])}
            \hlstd{solution[param2change]} \hlkwb{<<-} \hlstd{pars}\hlopt{$}\hlstd{ub[param2change]}

        \hlstd{\}}

        \hlstd{ObjValSol} \hlkwb{<<-} \hlkwd{goal}\hlstd{(solution)}
        \hlstd{FitnessSol} \hlkwb{<<-} \hlkwd{evaluation}\hlstd{(ObjValSol)}

        \hlcom{# a greedy selection is applied between the current solution i and its mutant*/}
        \hlkwa{if} \hlstd{(FitnessSol}\hlopt{>}\hlstd{fitness[i])}
        \hlstd{\{}
          \hlcom{# If the mutant solution is better than the current solution i, replace the solution with the mutant and reset the trial counter of solution i*/}
          \hlstd{trial[i]} \hlkwb{<<-} \hlnum{0}
          \hlstd{Foods[i,]} \hlkwb{<<-} \hlstd{solution}

          \hlstd{f[i]}\hlkwb{<<-}\hlstd{ObjValSol}
          \hlstd{fitness[i]}\hlkwb{<<-}\hlstd{FitnessSol}
        \hlstd{\}} \hlcom{#if the solution i can not be improved, increase its trial counter*/}
        \hlkwa{else} \hlstd{trial[i]} \hlkwb{<<-} \hlstd{trial[i]}\hlopt{+}\hlnum{1}
      \hlstd{\}}
      \hlstd{i} \hlkwb{<-} \hlstd{i} \hlopt{+} \hlnum{1}
      \hlkwa{if} \hlstd{(i}\hlopt{==}\hlstd{pars}\hlopt{$}\hlstd{FoodNumber) i} \hlkwb{<-} \hlnum{1}
      \hlcom{# end of onlooker bee phase}
    \hlstd{\}}
    \hlkwd{return}\hlstd{(Foods)}
  \hlstd{\}}

  \hlcom{# determine the food sources whose trial counter exceeds the "limit" value.}
  \hlcom{# In Basic ABC, only one scout is allowed to occur in each cycle*/}

  \hlstd{SendScoutBees} \hlkwb{<-} \hlkwa{function}\hlstd{() \{}
    \hlstd{maxtrialindex} \hlkwb{<-} \hlnum{1}
    \hlkwa{for} \hlstd{(i} \hlkwa{in} \hlnum{1}\hlopt{:}\hlstd{pars}\hlopt{$}\hlstd{FoodNumber) \{}
      \hlkwa{if} \hlstd{(trial[i]} \hlopt{>} \hlstd{trial[maxtrialindex]) maxtrialindex} \hlkwb{<-} \hlstd{i}
    \hlstd{\}}

    \hlkwa{if} \hlstd{(trial[maxtrialindex]} \hlopt{>=} \hlstd{pars}\hlopt{$}\hlstd{limit)} \hlkwd{init}\hlstd{(maxtrialindex)}
  \hlstd{\}}


 \hlstd{selection} \hlkwb{<-} \hlkwa{function}\hlstd{(}\hlkwc{history}\hlstd{,} \hlkwc{model}\hlstd{)\{}
   \hlstd{sel} \hlkwb{<-} \hlkwd{SendOnlookerBees}\hlstd{()}
   \hlkwd{return}\hlstd{(sel)}
 \hlstd{\}}

  \hlstd{initModel} \hlkwb{<-} \hlkwa{function}\hlstd{(}\hlkwc{history}\hlstd{)\{}
    \hlkwd{MemorizeBestSource}\hlstd{()}
    \hlkwd{SendEmployedBees}\hlstd{()}
    \hlkwd{evaluateList}\hlstd{(}\hlkwa{NULL}\hlstd{,} \hlkwa{NULL}\hlstd{)}
    \hlkwd{return}\hlstd{(GlobalParams)}
  \hlstd{\}}


  \hlstd{modelUpdate} \hlkwb{<-} \hlkwa{function}\hlstd{(}\hlkwc{selectedPoints}\hlstd{,} \hlkwc{oldModel}\hlstd{)\{}
    \hlkwd{MemorizeBestSource}\hlstd{()}
    \hlkwd{return}\hlstd{(GlobalParams)}
  \hlstd{\}}

  \hlstd{variation} \hlkwb{<-} \hlkwa{function}\hlstd{(}\hlkwc{selectedPoints}\hlstd{,} \hlkwc{newModel}\hlstd{)\{}
    \hlkwd{SendScoutBees}\hlstd{()}
    \hlkwd{SendEmployedBees}\hlstd{()}
    \hlkwd{return}\hlstd{(Foods)}
  \hlstd{\}}
  \hlcom{#push a LIST of points into the history}
  \hlstd{historyPush}\hlkwb{<-}\hlkwa{function}\hlstd{(}\hlkwc{oldHistory}\hlstd{,} \hlkwc{newPoints}\hlstd{)}
  \hlstd{\{}
    \hlcom{#    newHistory<-c(oldHistory,newPoints)}
    \hlcom{#    return (newHistory)}
    \hlkwd{return}\hlstd{(Foods)}
  \hlstd{\}}
  \hlcom{#read a LIST of points pushed recently into the history}
  \hlstd{historyPop}\hlkwb{<-}\hlkwa{function}\hlstd{(}\hlkwc{history}\hlstd{,} \hlkwc{number}\hlstd{)}
  \hlstd{\{}
    \hlstd{stop}\hlkwb{=}\hlkwd{length}\hlstd{(history)}
    \hlstd{start}\hlkwb{=}\hlkwd{max}\hlstd{(stop}\hlopt{-}\hlstd{number}\hlopt{+}\hlnum{1}\hlstd{,}\hlnum{1}\hlstd{)}
    \hlkwd{return}\hlstd{(history[start}\hlopt{:}\hlstd{stop])}
  \hlstd{\}}

  \hlstd{aggregatedOperator}\hlkwb{<-}\hlkwa{function}\hlstd{(}\hlkwc{history}\hlstd{,} \hlkwc{oldModel}\hlstd{)}
  \hlstd{\{}
    \hlstd{selectedPoints}\hlkwb{<-}\hlkwd{selection}\hlstd{(history, oldModel)}
    \hlstd{newModel}\hlkwb{<-}\hlkwd{modelUpdate}\hlstd{(selectedPoints, oldModel)}
    \hlstd{newPoints}\hlkwb{<-}\hlkwd{variation}\hlstd{(selectedPoints, newModel)}
    \hlkwd{return} \hlstd{(}\hlkwd{list}\hlstd{(}\hlkwc{newPoints}\hlstd{=newPoints,}\hlkwc{newModel}\hlstd{=newModel))}
  \hlstd{\}}

  \hlstd{p} \hlkwb{<-} \hlnum{0}
  \hlstd{history} \hlkwb{<-} \hlkwd{initialize}\hlstd{(Foods)} \hlcom{#return initialized foods - points which bees will estimate}
  \hlstd{model} \hlkwb{<-} \hlkwd{initModel}\hlstd{(history)}

  \hlstd{iter} \hlkwb{<-} \hlnum{0}
  \hlkwa{while} \hlstd{((iter} \hlkwb{<-} \hlstd{iter} \hlopt{+} \hlnum{1}\hlstd{)} \hlopt{<} \hlstd{pars}\hlopt{$}\hlstd{maxCycle} \hlopt{&&} \hlstd{p} \hlopt{<=} \hlstd{pars}\hlopt{$}\hlstd{critical)}
  \hlstd{\{}
    \hlstd{aa} \hlkwb{<-} \hlkwd{aggregatedOperator}\hlstd{(history, model)}
    \hlstd{aa}\hlopt{$}\hlstd{newPoints} \hlkwb{<-} \hlkwd{evaluateList}\hlstd{(aa}\hlopt{$}\hlstd{newPoints, model)}
    \hlstd{history} \hlkwb{<-} \hlkwd{historyPush}\hlstd{(history, aa}\hlopt{$}\hlstd{newPoints)}
    \hlstd{model} \hlkwb{<-} \hlstd{aa}\hlopt{$}\hlstd{newModel}
  \hlstd{\}}

  \hlkwd{return}\hlstd{(}
    \hlkwd{list}\hlstd{(}
      \hlkwc{x}\hlstd{=GlobalParams,}
      \hlkwc{y}\hlstd{=}\hlkwd{goal}\hlstd{(GlobalParams)}
    \hlstd{)}
  \hlstd{)}
\hlstd{\}}
\end{alltt}
\end{kframe}
\end{knitrout}

The R session information (including the OS info, R version and all
packages used):

\begin{knitrout}
\definecolor{shadecolor}{rgb}{0.969, 0.969, 0.969}\color{fgcolor}\begin{kframe}
\begin{alltt}
\hlkwd{sessionInfo}\hlstd{()}
\end{alltt}
\begin{verbatim}
## R version 3.2.2 (2015-08-14)
## Platform: x86_64-w64-mingw32/x64 (64-bit)
## Running under: Windows 8 x64 (build 9200)
## 
## locale:
## [1] LC_COLLATE=English_United States.1252  LC_CTYPE=English_United States.1252   
## [3] LC_MONETARY=English_United States.1252 LC_NUMERIC=C                          
## [5] LC_TIME=English_United States.1252    
## 
## attached base packages:
## [1] stats     graphics  grDevices utils     datasets  methods   base     
## 
## other attached packages:
## [1] knitr_1.12.3 alhe_0.1    
## 
## loaded via a namespace (and not attached):
## [1] magrittr_1.5  formatR_1.2.1 tools_3.2.2   stringi_1.0-1 highr_0.5.1   stringr_1.0.0
## [7] evaluate_0.8
\end{verbatim}
\begin{alltt}
\hlkwd{Sys.time}\hlstd{()}
\end{alltt}
\begin{verbatim}
## [1] "2016-01-22 13:14:23 CET"
\end{verbatim}
\end{kframe}
\end{knitrout}


\end{document}
